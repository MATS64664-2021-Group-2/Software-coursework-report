
\subsection{Parameters calculation}


\noindent
RHF corresponds to the weight average of hydride length multiplied by a weighting factor. The formula employed is described below:


\begin{equation} \label{RHF_eqn}
RHF =  \frac{\sum_i L_i f_i }
            {\sum_i L_i}
\end{equation}


\noindent
Where $L_i$ is the length of each hydride and $f_i$ is the weighting factor, a parameter that has different values depending on the hydrides orientation. Hydrides with orientations between 0-40° to the transverse direction have a $f_i$ of 0, when the orientation is between 40° and 65°, the $f_i$ is 0.5, and hydrides with an orientation of 65–90° have a $f_i$ of 1 \cite{COLAS2013586}. 

\noindent
To calculate the HCC, a rectangular area in the micrograph of the alloy is separated, and the length of each radial hydride in that area is measured. The formula applied is shown below:

\begin{equation} \label{HCC_eqn}
HCC =  \frac{HC_1 + HC_2 + HC_3... }
            {L}
\end{equation}

\noindent
Where $HC_i$ is the length of each radial hydride and L is the height of the selected rectangle to measure \cite{SIMON2021152817}.

\noindent
To calculate the RHCF, the length of each radial hydride in a section of the cladding is measured. This time, the section will have a length of 150 µm along the arc length of the cladding, and the width will be all the cladding thickness. The formula applied in this case is:

\begin{equation} \label{RHCF_eqn}
RHCF =  \frac{max (L_1 + L_2 + L_3...) }
            {h_m}
\end{equation}

\noindent
Where $L_i$ is the length of each radial hydride within the 150 µm of arc length, and $h_m$ is the cladding thickness \cite{SIMON2021152817}.

In Figure \ref{fig:parameters_drawing}, there is a drawing where the calculations of the HCC (a) parameter and the RHCF parameter (b) are illustrated.

\begin{figure}[h] %  figure placement: here, top, bottom, or page
    \centering
    \includegraphics[width=5.5in]{Figures/2-Parameters/parameters_drawing.png}
    \caption{Drawing of hydride platelets, showing how to calculate the parameters a) HCC, b) RHCF \cite{SIMON2021152817}.}
    \label{fig:parameters_drawing}
\end{figure}