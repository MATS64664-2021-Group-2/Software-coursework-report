%========= Literature review
\section{Literature review}

A.T. Motta et al. \cite{MOTTA2019440} stated that, as hydrides orientation has a significant impact on the mechanical response of the cladding material, the modelling of hydride precipitation and dissolution should cover not only the volume fraction of the precipitated hydrides, but also their morphology. They said that the presence of radial hydride particles has a direct influence on crack propagation, since it provides an energetically favourable fracture path through the cladding wall, whereas circumferential hydrides interferes to a small extent. However, when different thermo-mechanical processes are applied, hydrides can evolve from a circumferential to radial orientation, which reduces the overall strength of the cladding. They pointed as an important research need to establish the limits at which the hydride microstructure created significantly affect cladding ductility.

\cite{COLAS2013586}
\cite{SHARMA2018546}
\cite{SUNIL2020152457}