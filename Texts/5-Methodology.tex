%========= Methodology
\section{Methodology \& Discussion}

In this work, three types of thresholding was performed: Otsu, K-means and Gaussian. The HCC was calculated for all of them to check which method is the most accurate in terms of hydrides connectivity evaluation. Using GitHub to collaborate, we developed a Jupyter notebook were many functions are called to show clarity in the script.

The following sections were developed throughout the workflow:

\begin{enumerate}

    \item \textbf{Import packages.}
This sections is used to call all the packages needed to the correct running of the whole script.

    \item\textbf{ Import the images from GitHub.}
In this part of the code, the images to analyse are loaded.

    \item \textbf{Image processing.}
Because of the large area of the images being analysed, there are shadows that may lead us to incorrect results. To solve this, here it is included some code to break the image into vertical grayscale strips. The strips are also blurred and saved in their own directories.

    \item \textbf{Thresholding.}
In this part, the grayscale and blurred strips are transformed into binary images and three different thresholding methods are applied to see which one offers the best result.
o	Otsu thresholding
o	K-means thresholding
o	Gauss thresholding

    \item \textbf{Connectivity of microstructure.}
In this section, the connectivity between hydrides in the radial direction is checked and the edges are detected to discover the contours within the slices. With this information, the HCC parameter of each strip is measured, and finally the average HCC value is obtained for each of the three thresholding methods.

    \item \textbf{Testing of functions.}
Testing the code in each function allows for a more robust execution of the code by integrating and automating the test procedure. Example micrographs created and modified with known will enable us to know what outputs to expect for each function's input. Comparing the similarity of images gives us a pass or fail criteria.\par

Description of test methods:
        \begin{itemize}
            \item Thresholding
            \item Connectivity
            \item Loading
            \item Parameters
            \item Plot
            \item Processing
        \end{itemize}

By anticipating how the end-user might use the code and its environment, testing decreases the likelihood of unexpected errors. 



\end{enumerate}


