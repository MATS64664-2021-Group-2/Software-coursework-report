
%%%%%%%%%%%%%%%%%%%%%%%%%%%%%%%%%%%%%%%%%%%%%%%%%%%%%%%%%%%%%%%%%
% This is the main text file, in which all the sections of this report are called and compiled

%=================== Beginning ==================

\documentclass[12pt]{article}
\usepackage{graphicx}
\usepackage{fontspec}
\usepackage[utf8]{inputenc}
\usepackage[english]{babel}
% Calibri Font
\setromanfont[
BoldFont=CalibriBold.ttf,
ItalicFont=CalibriItalic.ttf,
BoldItalicFont=CalibriBoldItalic.ttf,
]{Calibri.ttf}

%=============== Margin
\usepackage{geometry}
\geometry{a4paper, left=22mm,  top=22mm, bottom=22mm, right=22mm}
%===============
\usepackage{setspace}
\usepackage[document]{ragged2e} % left-alignment
\hyphenpenalty=10000
\tolerance=10000
%--------------------------------
\usepackage{amsmath} % Mathematical Equations 
\usepackage{subcaption}
\usepackage{caption}

\begin{document}
\onehalfspacing

%===================== Cover =======================

\thispagestyle{empty} 
\begin{titlepage}
    \centering
    \includegraphics[width=0.25\textwidth]{Figures/Logos/uom_logo.pdf}
    \hspace{170 em}
    \includegraphics[width=0.35\textwidth]{Figures/Logos/Sheffield.pdf}
    \begin{center}
       \vspace*{4cm}
       {\LARGE Research Software Engineering Practice}
       \vspace{3cm}
    \begin{large}   
    

         
         \vspace{0.5cm}

        {\LARGE Analysis of the connectivity of hydrides within the microstructure of Zr alloys} \\

       \vspace{1.5cm}
        
        {\bf \today} \\
                
        
       \vspace{3 cm}
        Group 2 \\
       \textbf{Laura Gonzalez,
       Wunmi Olukoya,
       Jamie McGregor, 
       Enn Veikesaar }\\

       \vfill
       \centering
       
        {\bf \large Advanced Metallic System CDT Program}\\
          
        \includegraphics[width=0.35\textwidth]{Figures/Logos/CDT.JPG}
    
    \end{large}  
   \end{center}
\end{titlepage}

%===================== Abstract ==================

\section*{Abstract}

\justifying
\noindent
The presence of hydrides is a major concern in Zr alloys due to their embrittling effect. It has been observed that extent of embrittlement depends on the hydrides disposition and connectivity in the alloy. Therefore, characterizing their connectivity would allow the creation of tougher Zr alloys for nuclear applications. In this project, a model has been developed to evaluate the hydrides connectivity on Zr alloy microstructures. The model, developed with Python language, consists of loading up existing micrographs of Zr alloy microstructures, binarising them using different thresholding methods (otsu, k-means and adaptive Gauss thresholding) and obtaining the Hydride Continuity Coefficient (HCC) to study the hydrides connectivity. 

%=================== Table of Content ======================

\newpage
\begin{singlespacing}
\tableofcontents
\end{singlespacing}
\setlength{\parskip}{1em}
\renewcommand{\baselinestretch}{2.0}


%================= Start of the report ===================

\newpage 
\pagenumbering{arabic}
\setcounter{page}{1}
\onehalfspacing



%=================== Contributions =======================

\newpage 
\section*{Contributions}

Description of each team members role and contribution

\begin{enumerate}
\item Laura Gonzalez:
    \begin{enumerate}
          \item second level item
          \item second level item
    \end{enumerate}
\item Wunmi Olukoya:
        \begin{enumerate}
          \item second level item
          \item second level item
    \end{enumerate}
\item Jamie McGregor: 
        \begin{enumerate}
          \item second level item
          \item second level item
    \end{enumerate}
\item Enn Veikesaar:
        \begin{enumerate}
          \item Gaussian Thresholding  
          \item Testing
    \end{enumerate}
\end{enumerate}

\clearpage
%====================== Sections ======================

\section{Introduction}

\subsection{The presence of hydrides in Zr-alloys}

\justify
\noindent
Zr alloys are widely employed as fuel cladding in the nuclear sector. When corrosion occurs in nuclear reactors, the released hydrogen penetrates into the cladding alloy. This material absorbs part of it but, once the solubility limit is reached, hydrogen precipitates into brittle hydrides platelets. Generally, the hydrides orient along the tube circumferential direction, but they may orient radially too. Cladding usually fails by a hoop stress, so radial hydrides will lead the alloy to fail earlier in deformation processes. Moreover, crack propagation through cladding thickness is alarming, and this occurrence will be especially facilitated by radial-oriented hydrides. As cladding alloys fail more easily under the presence of radial hydrides, it is important to study the changes in hydride orientation to control cladding embrittlement \cite{SIMON2021152817, COLAS2013586, SHARMA2018546, SUNIL}.

\noindent
Cladding failure under hoop stress is strongly  affected by these three factors:

\vspace{0.1 mm}
- Hydrogen and hydride content.

\vspace{0.1 mm}
- Fraction of radially-oriented hydrides.

\vspace{0.1 mm}
- Continuity in the hydrides along the thickness of the cladding \cite{SIMON2021152817}.

\noindent
To measure these factors, parameters such as the Radial Hydride Fraction (RHF), the Hydride Continuity Coefficient (HCC) and the Radial Hydride Continuity Factor (RHCF) are used. The RHF, always between 0 and 1, represents the fraction of radially-oriented hydrides. Higher values of this parameter are related to more propagation of the cracks through the cladding thickness. But some microstructures with different hydrides locations may have the same RHF value in some cases, as it is shown in Figure \ref{fig:RHF_comparison}. Therefore, it is also necessary to measure the continuity of the hydrides, as this variable has an important effect on crack propagation. The HCC and RHCF determine how close the hydride platelets are to each other and consider radial hydrides alignment across the cladding thickness, which is related to cracking propagation through the cladding thickness. These parameters, generally between 0 and 1, the higher they are, the more connected the hydrides across the thickness of the material will be \cite{SIMON2021152817}. Low values of HCC means that there are few hydrides and/or the hydrides are oriented along the circumferential direction \cite{SHARMA2018546}.

\vspace{50 mm}

\begin{figure}[h] %  figure placement: here, top, bottom, or page
    \centering
    \includegraphics[width=4.3in]{Figures/1-Introduction/same RHF.png}
    \caption{Same hydrides located differently: The same RHF value but different alignment \cite{SIMON2021152817}.}
    \label{fig:RHF_comparison}
\end{figure}

\vspace{0.1 in}

\subsection{Parameters calculation}


\noindent
RHF corresponds to the weight average of hydride length multiplied by a weighting factor. The formula employed is described below:


\begin{equation} \label{RHF_eqn}
RHF =  \frac{\sum_i L_i f_i }
            {\sum_i L_i}
\end{equation}


\noindent
Where $L_i$ is the length of each hydride and $f_i$ is the weighting factor, a parameter that has different values depending on the hydrides orientation. Hydrides with orientations between 0-40° to the transverse direction have a $f_i$ of 0, when the orientation is between 40° and 65°, the $f_i$ is 0.5, and hydrides with an orientation of 65–90° have a $f_i$ of 1 \cite{COLAS2013586}. 

\noindent
To calculate the HCC, a rectangular area in the micrograph of the alloy is separated, and the length of each radial hydride in that area is measured. The formula applied is shown below:

\begin{equation} \label{HCC_eqn}
HCC =  \frac{HC_1 + HC_2 + HC_3... }
            {L}
\end{equation}

\noindent
Where $HC_i$ is the length of each radial hydride and L is the height of the selected rectangle to measure \cite{SIMON2021152817}.

\noindent
To calculate the RHCF, the length of each radial hydride in a section of the cladding is measured. This time, the section will have a length of 150 µm along the arc length of the cladding, and the width will be all the cladding thickness. The formula applied in this case is:

\begin{equation} \label{RHCF_eqn}
RHCF =  \frac{max (L_1 + L_2 + L_3...) }
            {h_m}
\end{equation}

\noindent
Where $L_i$ is the length of each radial hydride within the 150 µm of arc length, and $h_m$ is the cladding thickness \cite{SIMON2021152817}.

In Figure \ref{fig:parameters_drawing}, there is a drawing where the calculations of the HCC (a) parameter and the RHCF parameter (b) are illustrated.

\begin{figure}[h] %  figure placement: here, top, bottom, or page
    \centering
    \includegraphics[width=5.5in]{Figures/2-Parameters/parameters_drawing.png}
    \caption{Drawing of hydride platelets, showing how to calculate the parameters a) HCC, b) RHCF \cite{SIMON2021152817}.}
    \label{fig:parameters_drawing}
\end{figure}
\vspace{0.1 in}
\subsection{Limitations of each parameter}

\noindent
The problem of the RHF is that it does not differentiate between hydrides oriented within the ranges 0-40°, 40-65°, and 65-90° \cite{SIMON2021152817}. Thus, microstructures with different radial hydrides orientations can have the same $f_i$, which will lead to obtain the same value of RHF.

\noindent
The continuity factors do not differentiate some situations either. For example, the HCC is the same in the three situations illustrated in Figure \ref{fig:limitationhcc}, which certainly facilitate the cracking propagation to different extents. In Figure \ref{fig:limitationrhcf} it can be seen that two different situations entail the same value of RHCF and, on the contrary, Figure \ref{fig:sameconnectivity} shows how two equivalent situations may lead to different values of RHCF \cite{SIMON2021152817}.

\begin{figure}[h] %  figure placement: here, top, bottom, or page
    \centering
    \includegraphics[width=4.5in]{Figures/3-Limitations/HCC comparison.png}
    \caption{Radial-oriented hydrides differently located but with the same HCC value.}
    \label{fig:limitationhcc}
\end{figure}

\begin{figure}[h] %  figure placement: here, top, bottom, or page
    \centering
    \includegraphics[width=4.5in]{Figures/3-Limitations/RHCF comparison 1.png}
    \caption{Radial-oriented hydrides with the same RHCF value but different connectivity.}
    \label{fig:limitationrhcf}
\end{figure}

\begin{figure}[h] %  figure placement: here, top, bottom, or page
    \centering
    \includegraphics[width=4.5in]{Figures/3-Limitations/RHCF comparison 2.png}
    \caption{Radial-oriented hydrides with the different RHCF value but the same connectivity.}
    \label{fig:sameconnectivity}
\end{figure}

\vspace{0.1 in}
%========= Literature review
\section{Literature review}

A.T. Motta et al. \cite{MOTTA2019440} stated that, as hydrides orientation has a significant impact on the mechanical response of the cladding material, the modelling of hydride precipitation and dissolution should cover not only the volume fraction of the precipitated hydrides, but also their morphology. They said that the presence of radial hydride particles has a direct influence on crack propagation, since it provides an energetically favourable fracture path through the cladding wall, whereas circumferential hydrides interferes to a small extent. However, when different thermo-mechanical processes are applied, hydrides can evolve from a circumferential to radial orientation, which reduces the overall strength of the cladding. They pointed as an important research need, to establish the limits at which the created hydride microstructure significantly affect cladding ductility.

R.K. Sharma et al. \cite{SHARMA2018546} studied the propagation of cracks in  Zr-2.5\%Nb samples with hydrides in their microstructure. By applying annealing, they decreased the value of HCC, and this improves the fracture resistance of the material. In Figure  \ref{fig:ref2}, it can be seen how the fracture toughness $(KJ_{max})$ increases when HCC is smaller, and how it reaches extremely high values when HCC is near 0.

\begin{figure}[h] %  figure placement: here, top, bottom, or page
    \centering
    \includegraphics[width=3.8in]{Figures/4-Lit. Review/HCCref2.png}
    \caption{Fracture toughness $(KJ_{max})$ at room temperature and transition temperature evolution with respect to HCC \cite{SHARMA2018546}.}
    \label{fig:ref2}
\end{figure}

S. Sunil et al. \cite{SUNIL2020152457} studied the Delayed Hydride Cracking (DHC) on the same alloy, Zr-2.5\%Nb. The material had a radial hydrides presence corresponding to a mean HCC value of around 0.8 at room temperature. After heating the samples at 200 and 225°C, their HCC value lowed to 0.576 and 0.44, respectively. DHC has an incubation period, a stable average crack growth velocity (DHC velocity) and a threshold stress intensity factor ($K_I$). In presence of radial hydrides, an increase of DHC velocity with $K_I$ factor was observed. They related this fact to the increase in hydride fracture zone volume, which is marked in dark blue in Figure \ref{fig:ref3} (an amplification of the hydride fracture zone can be observed on the right).

\begin{figure}[h] %  figure placement: here, top, bottom, or page
    \centering
    \includegraphics[width=4.3in]{Figures/4-Lit. Review/propagation.JPG}
    \caption{Crack propagation a) In presence of longitudinal hydrides (planar propagation), b and c) In presence of radial hydrides, before and after crack propagation (zig-zag propagation)  \cite{SUNIL2020152457}.}
    \label{fig:ref3}
\end{figure}


\cite{COLAS2013586}

\cite{SUNIL2020152457}
\vspace{0.1 in}
%========= Methodology
\section{Methodology \& Discussion}

In this work, three types of thresholding was performed: Otsu, K-mean and Gaussian. The HCC was calculated for all of them to check which method is the most accurate in terms of hydrides connectivity evaluation. Using GitHub to collaborate, we developed a Jupyter notebook were many functions are called to show clarity in the script.

The following sections were developed throughout the workflow:

\begin{enumerate}

    \item \textbf{Import packages.}
This sections is used to call all the packages needed to the correct running of the whole script.

    \item\textbf{ Import the images from GitHub.}
In this part of the code, the images to analyse are loaded.

    \item \textbf{Image processing.}
Because of the large area of the images being analysed, there are shadows that may lead us to incorrect results. To solve this, here it is included some code to break the image into vertical grayscale strips. The strips are also blurred and saved in their own directories.

    \item \textbf{Thresholding.}
In this part, the grayscale and blurred strips are transformed into binary images and three different thresholding methods are applied to see which one offers the best result.
o	Otsu thresholding
o	K-means thresholding
o	Gauss thresholding

    \item \textbf{Connectivity of microstructure.}
In this section, the connectivity between hydrides in the radial direction is checked and the edges are detected to discover the contours within the slices. With this information, the HCC parameter of each strip is measured, and finally the average HCC value is obtained for each of the three thresholding methods.
\end{enumerate}

\section{Results \& Discussion}

\vspace{0.1 in}
%========= Methodology
\section{Python Implementation}

For this report, python was chosen as the coding language. While this choice is largely one of personal preference and accessibility, within the confines of this project collectively our group was supplied with training in python and the advantage of an open source system compared to a licensed system is not one to be ignored. Following along the same path as the methodology, the code was implemented in relation to an image processing, image operation and finally data analysis goal setting.

\subsection{Image Processing}
Briefly consisting of reading the image, splitting the image and binarising the image, this step aims to robustly prepare an input of unknown scale and composition before being passed onto thresholding.

\noindent
To achieve this result, the OpenCV and NumPy libraries were imported. With use of the reading and writing image structures the images were able to be transferred from their original jpg files into 2D arrays containing the greyscale values of their pixels as their representative cells (Fig.\ref{fig:ReadingCode}). From this point onwards, to enhance quality of images passed between functions and to the user, all image files will be saved as a lossless png format.

\begin{figure}[h] %  figure placement: here, top, bottom, or page
    \centering
    \includegraphics[width=6in]{Figures/Code/Reading an image.png}
    \caption{Screenshot of the code for defining a directoy containing micrographs and reading an image with OpenCV}
    \label{fig:ReadingCode}
\end{figure}


\noindent
Now that the image has been transformed into a form of data manipula
\vspace{0.1 in}
\section{Results \& Discussion}

In this section, the results from the implementation of the code for a specific example are shown. The image that was analysed as an example, as well as its imported and processed states, are shown in Figure \ref{fig:load&processing}.

\begin{figure}[h] %  figure placement: here, top, bottom, or page
    \centering
    \includegraphics[width=1.8in]{Figures/5-Results/chu7.jpg}
    \includegraphics[width=1.8in]{Figures/5-Results/loading.PNG}
    \includegraphics[width=1.8in]{Figures/5-Results/processing.PNG}
    \caption{Image to be analysed: a) Original state, b) After loading, c) After processing.}
    \label{fig:load&processing}
\end{figure}

\noindent
In Figure \ref{fig:thresholding}, it can be observed the image after the three different threshold were applied: Otsu, K-means and Gauss. The K-means threshold seems to be the most clear of them so far.

\begin{figure}[h] %  figure placement: here, top, bottom, or page
    \centering
    \includegraphics[width=1.8in]{Figures/5-Results/ThOtsu.PNG}
    \includegraphics[width=1.8in]{Figures/5-Results/ThKmeans.PNG}
    \includegraphics[width=1.8in]{Figures/5-Results/ThGauss.PNG}
    \caption{Images after three different thresholds were applied: a) Otsu b) K-means, c) Gauss.}
    \label{fig:thresholding}
\end{figure}

\noindent
In Figure \ref{fig:connectivity}, it can be observed the resulting images after studying the connectivity between hydrides in the radial direction. Due to the inconsistent quality of thresholding obtained by the Gaussian method, the image obtained from the Gauss thresholding was excluded from HCC analysis. The other two images were subjected to the HCC analysis. 

Two methods were followed to calculate the HCC value. With the first one, the HCC value is calculated for every single pixel column and all the numbers are averaged to give a resulting HCC value. This method was discarded because the results (0.11 for Otsu thresholding and 0.13 for K-means thresholding) were too low for their corresponding connectivity images (Fig. \ref{fig:connectivity} a and b, respectively). The images show a big amount of radial hydrides, so it is expectable to obtain a higher value of HCC.

With the second method, the HCC value of every slice is measured, and then the values are averaged. The results of the second method fit better their corresponding images. The HCC value for the image Otsu-thresholed is the same as for the K-means thresholded: 0.65. 

\begin{figure}[h] %  figure placement: here, top, bottom, or page
    \centering
    \includegraphics[width=1.8in]{Figures/5-Results/Otsu.PNG}
    \includegraphics[width=1.8in]{Figures/5-Results/Kmeans.PNG}
    \includegraphics[width=1.8in]{Figures/5-Results/Gauss.PNG}
    \caption{Connectivity of the images with a) Otsu threshold, b) K-means threshold, c) Gauss threshold.}
    \label{fig:connectivity}
\end{figure}

This model was used in different images for comparison. The images analysed and their corresponding HCC values obtained after Otsu and K-means thresholdings are shown in Figure \ref{fig:compa}. As can be expected, the highest values of HCC are obtained for the microstructure showed in Fig 12a, which has many hydrides in both directions, longitudinal and radial, and therefore, the connectivity is favoured. When the majority of the hydrides are longitudinal, as in Fig 12b, the HCC value is lower. If we compare Fig12 a and c, the main difference between these microstructures is the amount of hydrides. The microstructure a has higher values of HCC than the microstructure c. Finally, the microstructure d shows less longitudinal hydrides than the microstructure b, so the HCC is also lower.


\begin{figure}[h] %  figure placement: here, top, bottom, or page
    \centering
    \includegraphics[width=5.5in]{Figures/5-Results/Comparison.PNG}
    \caption{HCC values for different microstructures: a) Many longitudinal and radial hydrides, b) Majority of longitudinal hydrides, c) Some longitudinal and radial hydrides, d) Some longitudinal hydrides.}
    \label{fig:compa}
\end{figure}
\vspace{0.1 in}
%========= Conclusions
\newpage
\section{Conclusions \& Future Work}

\noindent
The HCC value is reduced when there are less hydrides in the microstructure.

\noindent
The HCC value is lower when there is a preferred direction in the shape of the hydrides.

\noindent
For future work, it would be interesting to introduce into the model the calculations of other parameters such as the RHF and the RHCF.

\newpage
\singlespacing
\bibliography{biblography}
\bibliographystyle{IEEEtran}

\end{document}